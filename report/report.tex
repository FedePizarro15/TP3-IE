\documentclass[12pt, a4paper]{article}
\usepackage[utf8]{inputenc}
\usepackage[T1]{fontenc}
\usepackage{lmodern}

\usepackage[spanish]{babel}

\usepackage{amsmath,amssymb}

\usepackage{booktabs}
\usepackage{longtable}
\usepackage{array}

\usepackage{geometry}
\geometry{left = 2.5cm, right = 2.5cm, top = 2.5cm, bottom = 2.5cm}

\usepackage{graphicx, subfig}

\usepackage{float}

\usepackage{hyperref}
\hypersetup{hidelinks=true}

\usepackage{xcolor}
\usepackage{caption}

\captionsetup[figure]{name=Fig.,labelsep=colon}
\captionsetup[table]{font = small, skip = 5pt}

\usepackage{indentfirst}

\usepackage{setspace}

\usepackage{fancyhdr}

\newcommand{\vectorformat}[1]{\overline{\mathbf{#1}}}

\pagestyle{fancy}
\fancyhf{}

\fancypagestyle{plain}{
    \fancyhf{}
    \fancyfoot[C]{Página~\thepage}
    \renewcommand{\headrulewidth}{0pt}
    \renewcommand{\footrulewidth}{1pt}
    }
    
    \fancyhead[L]{\textit{Localización mediante triangulación}}
    \fancyhead[R]{\textit{Orozco, Pizarro}}
    
    \fancyfoot[C]{Página~\thepage}
    
    \renewcommand{\headrulewidth}{1pt}
    \renewcommand{\footrulewidth}{1pt}
    
  	\title{Localización mediante triangulación\\[2ex]\large Trabajo Práctico N°2 - Grupo 13\\[2ex]Inferencia y Estimación (I206)}
    \author{Alejandro Orozco Lopez, Federico Pizarro Dal Maso\\[1ex]\small{36625 - \href{mailto:aorozcolopez@udesa.edu.ar}{aorozcolopez@udesa.edu.ar}, 36584 - \href{mailto:fpizarrodalmaso@udesa.edu.ar}{fpizarrodalmaso@udesa.edu.ar}}}
    
\begin{document}
\captionsetup[subfloat]{captionskip=0pt}

\setlength{\parskip}{1em}

\captionsetup[table]{name=Tabla}

\maketitle

\begin{center}
    {\includegraphics[width=0.3\textwidth]{udesa_logo.png}}
\end{center}

\begin{abstract}
Este trabajo presenta la implementación y análisis de un sistema de localización por triangulación resuelto mediante el método de regresión lineal por Cuadrados Mínimos. Se desarrolló un modelo matemático linealizando las relaciones trigonométricas entre las referencias y el objetivo. Mediante simulaciones, se evaluó la sensibilidad del estimador frente a ruido gaussiano, demostrando su insesgadez estadística. Además, se estudió el impacto de la geometría de la red a través del número de condición, identificando que las disposiciones colineales generan inestabilidad numérica y amplificación del error. Finalmente, la validación experimental en campo corroboró los resultados teóricos, logrando estimar la posición con un error de 1.96 metros gracias a una configuración geométrica bien condicionada.
\end{abstract}

\newpage

\section*{Introducción}

En el presente trabajo práctico, nos enfocamos en localizar un objetivo por medio del método de la triangulación, utilizando mediciones angulares desde múltiples referencias, a las cuales denominaremos anclas, para estimar la posición de un punto desconocido. Dado que las mediciones reales siempre presentan incertidumbre, el sistema se plantea como un problema de regresión lineal sobredeterminado, el cual se resuelve mediante el método de Cuadrados Mínimos (Least Squares) para obtener una estimación óptima.

Consideramos un escenario en el plano cartesiano donde se desea hallar la posición desconocida de un objetivo, denotada como el vector $\mathbf{p} = (x, y)$.

Para ello, disponemos de $M$ puntos de referencia o ``anclas'' con posiciones conocidas, denotadas como $\mathbf{p}_i = (x_i, y_i)$ para $i = 1, \dots, M$. Además, desde la posición del objetivo se miden los ángulos $\theta_i$ relativos a cada ancla.

En este contexto, es fundamental distinguir la naturaleza de las variables. Los datos obtenidos por mediciones son los ángulos ($\theta_i$), constituyendo las variables observadas. Por otro lado, buscamos determinar las coordenadas $x$ e $y$, siendo estas las incógnitas del sistema.

Dado que el objetivo es estimar los parámetros del modelo partiendo de las observaciones, estamos frente a un problema inverso. A diferencia de un problema directo, donde se calculan los datos teóricos (ángulos) a partir de una posición conocida, en este caso debemos inferir la posición que mejor explique los datos recolectados.

Asumimos que los ángulos $\theta_i$ se miden desde una dirección de referencia vertical (Eje Y) en sentido horario hacia la línea de visión que une el objetivo con el ancla $i$.

Bajo esta configuración geométrica, la relación trigonométrica entre las coordenadas y el ángulo medido está dada por la tangente del ángulo, que relaciona el cateto opuesto (diferencia en $x$) con el cateto adyacente (diferencia en $y$):

\[
  \tan(\theta_i) = \frac{x_i - x}{y_i - y}
\]

La ecuación anterior no es lineal respecto a las incógnitas $x$ e $y$ debido al cociente. Para aplicar técnicas de regresión lineal matricial, es necesario linealizar la expresión. Reordenando la ecuación obtenemos:

\begin{equation}
  \label{eq_linealizada}
  x - y \cdot \tan(\theta_i) = x_i - y_i \cdot \tan(\theta_i)
\end{equation}

Esta ecuación lineal describe la restricción que impone una sola medición $i$ sobre la posición del objetivo.

Al disponer de $M$ anclas, obtenemos un sistema de ecuaciones lineales que puede expresarse en forma matricial $\mathbf{X}\beta = \mathbf{Y}$.
Siendo el vector de incógnitas ($\beta \in \mathbb{R}^{2 \times 1}$), el cual contiene las coordenadas a estimar, la matriz de diseño ($\mathbf{X} \in \mathbb{R}^{M \times 2}$) contiene los coeficientes geométricos derivados de los ángulos y el vector de observaciones ($\mathbf{Y} \in \mathbb{R}^{M \times 1}$) contiene los términos independientes calculados a partir de las posiciones de las anclas y los ángulos.

El sistema completo para $M$ mediciones es:

\begin{equation}
\label{system_matrix}
\begin{bmatrix}
1 & -\tan(\theta_1) \\
1 & -\tan(\theta_2) \\
\vdots & \vdots \\
1 & -\tan(\theta_M)
\end{bmatrix}
\cdot
\begin{bmatrix}
x \\
y
\end{bmatrix}
=
\begin{bmatrix}
x_1 - y_1 \tan(\theta_1) \\
x_2 - y_2 \tan(\theta_2) \\
\vdots \\
x_M - y_M \tan(\theta_M)
\end{bmatrix}
\end{equation}

Generalmente, el número de anclas es $M \geq 3$, lo que resulta en un sistema sobredeterminado que no tiene solución exacta debido a errores de medición. Buscamos entonces el vector $\beta$ que minimice la norma del error al cuadrado $||\mathbf{X}\beta - \mathbf{Y}||^2$.

La solución analítica de Cuadrados Mínimos se obtiene mediante la ecuación:

\begin{equation}
  \label{least_squares_solution}
  \beta = (\mathbf{X}^T \mathbf{X})^{-1} \mathbf{X}^T \mathbf{Y}
\end{equation}

Esta expresión será la base algorítmica utilizada en el desarrollo experimental de este informe para estimar la posición del objetivo.

\newpage
\section*{Desarrollo Experimental}

Para cada una de las distintas configuraciones tuvimos que extraer las posiciones de cada ancla y los ángulos medidos, con los cuales pudimos armar para cada observacion una ecuación lineal como se presenta en la ecuación \ref{eq_linealizada}.

Se desarrolló la función \texttt{solve\_ls} para resolver el sistema lineal sobredeterminado derivado del modelo geométrico descrito en la introducción. Esta función es nuestra base para la implementación del método de triangulación.

A partir de las coordenadas de las anclas y los ángulos medidos, el algoritmo construye la matriz de diseño y el vector de observaciones según lo establecido en la ecuación \ref{system_matrix}. La matriz de diseño contiene en cada fila los coeficientes derivados de los ángulos medidos, mientras que el vector de observaciones se calcula combinando las posiciones de las anclas con las tangentes de los ángulos correspondientes. El sistema se resuelve aplicando directamente la solución analítica de Cuadrados Mínimos vista en la ecuación \ref{least_squares_solution}.

Para evaluar la robustez del estimador bajo condiciones realistas de medición, se implementó la función \texttt{noisy\_estimations}, que permite simular el efecto de la incertidumbre en las posiciones de las anclas.

Se realizó una simulación con $M=50$ iteraciones independientes. En cada iteración, las posiciones a las posiciones de las distintas anclas se les añadió un ruido con distribución $\mathcal{N}\,(\mu = 0, \sigma^2)$, simulando errores aleatorios de medición. Este ruido afecta tanto las coordenadas en $x$ como en $y$ de cada ancla de manera independiente.

Se evaluaron tres escenarios de incertidumbre correspondientes a varianzas $\sigma^2 \in \{4, 25, 100\}$, representando niveles crecientes de error en las mediciones. Para cada escenario, se registraron todas las estimaciones obtenidas y se calcularon estadísticas descriptivas para caracterizar la distribución del error.

Adicionalmente, se graficaron elipses de confianza con un intervalo del 95\%, calculadas a partir de la matriz de covarianza empírica de la nube de puntos estimada. Estas elipses permiten visualizar la dispersión bidimensional del error y su dirección principal, proporcionando una representación geométrica intuitiva de la incertidumbre en la estimación.

Para analizar el condicionamiento de la matriz de diseño, se implementó la función \texttt{condition\_number} utilizando la norma espectral (norma-2). Este indicador cuantifica cuánto se pueden amplificar los errores de entrada en la solución final: un número de condición menor a 1 indica un problema bien condicionado, mientras que valores grandes sugieren inestabilidad numérica.

Se estableció un umbral empírico de $\kappa > 10$ para clasificar una configuración como ``mal condicionada'', basandonos en la siguiente relación: 
\begin{equation}
    \text{Dígitos perdidos} \approx \log_{10}(\kappa)
\end{equation}

Las configuraciones con números de condición superiores a este umbral se consideran propensas a amplificación significativa de errores.

Se analizaron las configuraciones \texttt{config\_2}, \texttt{config\_3} y \texttt{config\_4} manteniendo una varianza de ruido fija ($\sigma^2=4$). Esto permitió evaluar cómo la ubicación de las anclas, independientemente del ruido de medición, afecta la estabilidad del sistema y la precisión alcanzable.

Para validar el método en condiciones realistas, se realizó un experimento de localización utilizando mediciones angulares desde múltiples puntos de referencia distribuidos en el campus universitario. Las posiciones de las anclas se midieron con GPS y los ángulos se obtuvieron mediante brújulas digitales.

\begin{figure}[h!]
\centering
\includegraphics[width=.85\linewidth]{campus.jpg}
\caption{Imagen del campus de la Universidad de San Andrés donde se realizó el experimento de localización.}
\label{fig:campus}
\end{figure}

El punto a estimar que elegimos se encontraba en el centro del círculo del campus y los puntos de referencia fueron el comedor, la entrada de la biblioteca y la entrada principal del edificio "Hirsch".

\newpage
\section*{Resultados y Análisis}

Para estimar la posición de $\mathbf{p}$, implementamos el algoritmo de Cuadrados Mínimos descrito en la ecuación \ref{least_squares_solution}. Se realizaron múltiples experimentos variando las configuraciones de anclas y los niveles de ruido aplicado a las mediciones angulares.

\begin{figure}[h!]
\centering
\includegraphics[width=.85\linewidth]{p_estimation.png}
\caption{Estimación de la posición $\beta$ con la configuración \texttt{config\_1}.}
\label{fig:p_estimation}
\end{figure}

Podemos observar en la Figura \ref{fig:p_estimation} la estimación obtenida para la configuración inicial (\texttt{config\_1}) sin ruido, donde el objetivo se encuentra en la posición real $\mathbf{p} = (100, 50)$. Esto confirma la certeza del modelo de cuadrados mínimos en condiciones ideales.

\newpage

Al incorporar ruido gausssiano con varianzas $\sigma^2 = 4, 25, 100$ a las posiciones de las anclas, se realizaron 50 estimaciones para cada nivel de ruido. Los resultados se presentan en las siguientes gráficas de dispersión:

\begin{figure}[H]
    \centering
    \subfloat[$\sigma^2 = 4$]{\includegraphics[width=0.45\textwidth]{noise_v_4.png}}\hfill
    \subfloat[$\sigma^2 = 100$]{\includegraphics[width=0.45\textwidth]{noise_v_100.png}}
    \caption{Gráficas de dispersión de las estimaciones $\beta$ bajo los  niveles de ruido $\sigma^2 = 4, 100$.}
    \label{fig:scatter_plots}
\end{figure}

Es claro apreciar cómo el aumento de la varianza del ruido afecta la dispersión de las estimaciones. El campo de incertidumbre se amplifica conforme incrementa $\sigma^2$, aunque la media de las estimaciones permanece cercana a la posición real del objetivo, evidenciando la insesgadez del estimador.

Luego, se analizaron las configuraciones \texttt{config\_2}, \texttt{config\_3} y \texttt{config\_4}, todas con un nivel de ruido fijo de $\sigma^2 = 4$. Los resultados obtenidos se resumen en la Tabla \ref{tab:condicionamiento}.

\begin{table}[h!]
\centering
\begin{tabular}{|c|c|c|}
\hline
\textbf{Configuración} & \textbf{Número de Condición ($\kappa$)} & \textbf{Estado del Problema} \\ \hline
\texttt{config\_2} & $7.77$ & Bien condicionado \\ \hline
\texttt{config\_3} & $17.7$ & Mal condicionado \\ \hline
\texttt{config\_4} & $701.73$ & Mal condicionado \\ \hline
\end{tabular}
\caption{Comparación del número de condición según la configuración de anclas.}
\label{tab:condicionamiento}
\end{table}

En la \texttt{config\_2}, obtuvimos un $\kappa \approx 7.77$. Al ser un valor bajo, el sistema es estable. La pérdida de precisión es manejable y las estimaciones se mantienen razonablemente cerca del valor real. Por el contrario, en las configuraciones 3 y 4, el número de condición se dispara y la imprecisión de las estimaciones aumenta considerablemente. 

\newpage

Analizando qué ocurre en las gráficas de dispersión para estas configuraciones mal condicionadas:

\begin{figure}[H]
    \centering
    \subfloat[\texttt{config\_2}]{\includegraphics[width=0.45\textwidth]{cond_2.png}}\hfill
    \subfloat[\texttt{config\_4}]{\includegraphics[width=0.45\textwidth]{cond_4.png}}
    \caption{Gráficas de dispersión de las estimaciones $\beta$ para configuraciones bien y mal condicionadas.}
    \label{fig:mal_conditioned}
\end{figure}

La causa de este mal condicionamiento radica en la disposición de las anclas. En los casos mal condicionados, las anclas se encuentran alineadas o angularmente muy próximas entre sí respecto al objetivo.

Esto provoca que las líneas de proyección trianguladas sean casi paralelas. Como consecuencia, la incertidumbre deja de distribuirse de forma circular y se ``aplana'' formando una elipse alargada a lo largo del eje de visión, como se evidencia en los gráficos de dispersión de las configuraciones 3 y 4. Una pequeña variación en el ángulo o posición del ancla desplaza drásticamente el punto de intersección (la solución) a lo largo de ese eje, degradando severamente la precisión del sistema.


Para validar el método en condiciones realistas, se realizó un experimento de localización utilizando mediciones angulares desde múltiples puntos de referencia distribuidos en el campus universitario. Las dos figuras siguientes muestran las coordenadas absolutas y las coordenadas centradas respecto a uno de los puntos de referencia.

\begin{figure}[H]
    \centering
    \subfloat[Coordenadas absolutas]{\includegraphics[width=0.45\textwidth]{exp.png}}\hfill
    \subfloat[Coordenadas centradas]{\includegraphics[width=0.45\textwidth]{exp_cent.png}}
    \caption{Localización experimental en entorno real.}
    \label{fig:real_experiment}
\end{figure}

Este experimento nos deja apreciar que la aplicación del método de Cuadrados Mínimos en un entorno real logra un resultado satisfactorio al momento de estimar la posición del objetivo. Aunque, debido a la inexactitud de la metodología usada para medir los datos, deja en evidencia la incertidumbre observacional. Sin embargo, como las referencias están distribuidas de forma dispersa en el plano, la precisión del modelo sigue siendo lo suficientemente certera para inferir la posición real, lo cual controla que la propagación del error no se amplifique significativamente, obteniendo un error de aproximadamente 1.96 metros. Además, con lo visto en los ejercicios anteriores, podemos describir a la matriz de diseño de este sistema como bien condicionada.

\newpage
\section*{Conclusión}

En este trabajo, se validó la eficacia del método de Cuadrados Mínimos para resolver el problema inverso de localización por triangulación, linealizando con éxito la relación trigonométrica entre las anclas y el objetivo. 

El análisis de sensibilidad demostró que el estimador es insesgado. Si bien el aumento en la varianza del ruido gaussiano incrementa la dispersión de las predicciones, la media de estas se mantiene centrada en la posición real, validando la estabilidad estadística del modelo ante errores de medición aleatorios.

Se identificó que la geometría de la red de anclas es el factor más crítico para la precisión. El análisis del número de condición reveló que configuraciones con anclas colineales o angularmente próximas generan sistemas mal condicionados. Esto provoca que la incertidumbre se aplane en elipses alargadas, amplificando drásticamente los errores en la dirección del eje de visión, haciendo que el sistema sea inutilizable en la práctica 

El experimento en el campus corroboró la teoría. A pesar de la incertidumbre inherente a las mediciones manuales con brújulas digitales, se logró una estimación con un error acotado de aproximadamente 1.96 metros. Este éxito se atribuye directamente a la selección de una configuración de anclas dispersa y bien condicionada, lo que evitó la amplificación del error observada en las simulaciones numéricas adversas.

\end{document}